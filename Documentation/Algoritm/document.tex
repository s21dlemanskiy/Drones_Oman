
 \documentclass[12pt]{article} % Документ принадлежит классу article, а также будет печататься в 12 пунктов.
\usepackage[russian]{babel} % Пакет поддержки русского языка
\usepackage{amsmath} %пакет формул
\title{Теория Алгоритма} % Заглавие документа
\usepackage{alltt}
\date{\today} % Дата создания
\parindent=1cm
 \begin{document}
 	\tableofcontents
 	\newpage
 	\section{Фурмолировка задачи}
 	
 	\section{Условия упрощения алгоритма}
 	\subsection{Фурмалировка условий упрощения алгоритма}
 	\hspace*{1cm}Так как мы пишем программу, что бы облегчить себе задачу, будем искать куда поставить почтамт методом переюора. Но перебор длжен быть организован так, что:\\
 	\hspace*{5mm}1. Точек куда можно поставить почтамт конечно, и их количество должно быть \(a<\text{n}<b\) ($10^5<n<10^6$).\\
 	\hspace*{5mm}2. Для каждой точки можно определить скалярную функцию входных данных(центры активности(далее ЦА), зоны запрета полета(далее NFZ), населенности в районе).
 	\subsection{Уточним первое условие}
 	\hspace*{10mm}Теперь определим \(a\), \(b\). Переменая $a$ отвечает за то что бы не упрастить задачу до одной точки. Я думаю $a$ должно быть таким, что растояние между точками меньше 100метров. Площадь Маската 3500 $km^2$ откуда получаем $a\approx \cfrac{35\cdot 10^8}{100^2} =35 \cdot 10^4   $, а если учесть что 1/3 маската это малонаселенные горы получаем $a \approx 10^5$. Значение $b$ можно определить из времени выполнения программы. Дадим напрмер на алгорим 10 минут. количество операций которое можно провести за это время можн опонять из такой программы:\\
 	\begin{alltt}
 		\textit{1 import datetime
 		2 t = datetime.datetime.now()
 		3 i = 0
 		4 while (datetime.datetime.now() - t).seconds < 30:
 		5     i += 1
 		6 print(i * 20)}
 	\end{alltt}
 	Вывод программы зависит от устройства на котором она запущена. У меня выдала $792801540$ или  примерно $8000 \cdot 10^5$ и если на одну точку брать хотя бы 800 простых операций получаем $b \approx10^6$ \par
 	Итак первое условие $10^5<n<10^6$
 	\subsection{Уточним второе условие}
 	\hspace*{10mm} Для начала определим еще один важный фактор. Наша программа раставляет пачтамты последовательно и от наиболее загруженного к наименее загруженому, отсюда получаем ситуацию: стоит почтамт и не так далеко находится ЦА. Мы конечно хотим поставить пачтамт рядом с ЦА, но тут уже стоит один. Отсюда следствие: функция должна учитывать так же уже поставленные пачтамты. \\
 	\hspace*{5mm}2.1 функция должна опираться на уже поставленные пачтамты \par
 	
 	
 \end{document}